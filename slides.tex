% \pdfpageattr {/Group << /S /Transparency /I true /CS /DeviceRGB>>}
\documentclass[10pt, xcolor={usenames, dvipsnames}]{beamer}

\usepackage[french, english]{babel}
\usepackage[utf8]{inputenc}
\usepackage[T1]{fontenc}
\usepackage{upgreek,textgreek}

\usepackage{animate}

\usepackage{ulem}

\usepackage{fontawesome}

\usepackage{changepage}

\usepackage{fnpct}

\usepackage{pifont}
\newcommand{\cmark}{\ding{51}}
\newcommand{\xmark}{\ding{55}}

\usepackage{amsmath}
\usepackage{amssymb}
\usepackage{bm}

\usepackage{graphicx}
\usepackage{hyperref}
\usepackage[style=alphabetic, autocite=inline, firstinits=true, maxnames=2]{biblatex}
\bibliography{bibliographie}
%\renewcommand*{\bibfont}{\small}

\usepackage{makecell}
\usepackage{tabularx, booktabs}
\usepackage{tikz}
\definecolor{light-gray}{gray}{0.65}
\tikzset{fade on/.code={\only<#1>{\color{light-gray}}}}
\tikzset{hide on/.code={\only<#1>{\color{white}}}}
\tikzset{bold on/.code={\only<#1>{\bfseries}}}
\tikzset{
  opinvisible/.style={opacity=0.2},
  visible on/.style={alt={#1{}{opinvisible}}},
  alt/.code args={<#1>#2#3}{%
    \alt<#1>{\pgfkeysalso{#2}}{\pgfkeysalso{#3}} % \pgfkeysalso doesn't change the path
  },
}

\usetikzlibrary{plotmarks,shapes,patterns}
\usepackage{colortbl}
\usepackage{ulem}
\usepackage{multicol}
\usepackage[overlay]{textpos}
\usepackage{multirow}
\usepackage{ragged2e}
\usepackage{rotating}
\usepackage{fancybox}
\usepackage{ulem}
\usepackage{overpic}
\usepackage{enumerate}
\usepackage{xfrac}
\usepackage{pgfplots}

\usepackage{transparent}

\usepackage{epstopdf}
\usepackage{epsfig}


\newcommand{\REF}{\textcolor{purple}{REF}}
\newcommand{\GO}[1]{\textcolor{blue}{#1}}
\newcommand{\YES}[1]{\textcolor{green}{#1}}
\newcommand{\NO}[1]{\textcolor{red}{#1}}

\newcommand{\BA}[1]{\textcolor{green!80!black!80}{#1}}
\newcommand{\BB}[1]{\textcolor{blue!80!black!80}{#1}}
\newcommand{\BP}[1]{\textcolor{purple!80!black!80}{#1}}
\newcommand{\BF}[1]{\textcolor{yellow!90!black!100}{#1}}

\usetheme{metropolis}

\title{Causal Broadcast: How to Forget?}
\author{Brice N\'edelec, Pascal Molli, and Achour Most{\'e}faoui}
\date{OPODIS'18, Hong Kong, December 17--19.}
\institute{
  \vspace{6em}
  \begin{center}
    \begin{minipage}{0.48\textwidth}
      \centering
      \includegraphics[width=0.5\textwidth]{logos/nantes.png}
    \end{minipage}
    \begin{minipage}{0.48\textwidth}
      \centering
      \includegraphics[width=0.3\textwidth]{logos/ls2n.jpg}
    \end{minipage}
  \end{center}
}



\begin{document}

\maketitle

\begin{frame}{Introduction}

  Causal broadcast is the core of many distributed applications such as
  distributed social networks, distributed collaborative software, or
  distributed data stores.

  \vspace{2.5em}
  
    \begin{minipage}{0.15\textwidth}
      \centering
      \includegraphics[width=0.7\textwidth]{logos/facebook.png}
    \end{minipage}
    \begin{minipage}{0.15\textwidth}
      \centering
      \includegraphics[width=0.7\textwidth]{logos/mastodon.png}
    \end{minipage}
    \begin{minipage}{0.15\textwidth}
      \centering
      \includegraphics[width=0.7\textwidth]{logos/telegram.png}
    \end{minipage}    
    \begin{minipage}{0.15\textwidth}
      \centering
      \includegraphics[width=0.7\textwidth]{logos/google.png}
    \end{minipage}
    \begin{minipage}{0.15\textwidth}
      \centering
      \includegraphics[width=0.8\textwidth]{logos/riak.png}
    \end{minipage}
    \begin{minipage}{0.15\textwidth}
      \centering
      \includegraphics[width=0.7\textwidth]{logos/wow.png}
    \end{minipage}

    \vspace{3em}  

    \textit{\textbf{What if\ldots} I want to develop fully decentralized
      applications?\ldots}

\end{frame}


\begin{frame}{Causal broadcast ensures causal order}

  Causal broadcast is a \textbf{reliable broadcast} that ensures a
  \textbf{specific ordering} among message deliveries.
  
  \vspace{1em}

  \begin{definition}[Causal order]
    The delivery order of messages follows the Lamport's happen before relationships
    $\rightarrow$ of the corresponding broadcasts.
    $\forall A,\,B,\,C,\, b_A(m) \rightarrow b_B(m') \implies d_C(m) \rightarrow
    d_C(m')$
  \end{definition}
  
  \begin{minipage}{0.49\textwidth}
    % \begin{center}
      \input{input/figtimelineA.tex}
    % \end{center}
  \end{minipage}~~
  \begin{minipage}{0.49\textwidth}
    \begin{center}
      
\begin{tikzpicture}

  \newcommand\X{40pt};
  \newcommand\Y{-20pt};

  \draw (0*\X, 0*\Y) node[above]{\faMale} circle (2pt);
  \draw (1*\X, 0*\Y) node[above]{\faFemale} circle (2pt);
  \draw (2*\X, 0*\Y) node[above]{\faChild}circle (2pt);

  \draw [->] (0*\X, -2+0*\Y) -- (0*\X, 5*\Y);
  \draw [->] (1*\X, -2+0*\Y) -- (1*\X, 5*\Y);
  \draw [->] (2*\X, -2+0*\Y) -- (2*\X, 5*\Y);

  \small

  \draw [->] (0*\X, 0.5*\Y) node[left]{b(\faImage)} -- (1*\X, 1.5*\Y)
  node[right]{d(\faImage)}; %% A B
  \draw [->] (0*\X, 0.5*\Y) -- (2*\X, 4.5*\Y)
  node[right]{d(\faImage) \NO{\xmark}}; %% A C

  \draw [->] (1*\X, 2*\Y) node[left]{b(\faCommentingO)} -- (2*\X, 3.5*\Y)
  node[right]{d(\faCommentingO) \NO{\xmark}}; %% B C
  \draw [->] (1*\X, 2*\Y) -- (0*\X, 3.5*\Y)
  node[left]{d(\faCommentingO)}; %% B A

  % \draw (2*\X, 4*\Y) node[right]{\NO{\xmark}};


\end{tikzpicture}
    \end{center}
  \end{minipage}


\end{frame}


\begin{frame}{Causal broadcast is a reliable broadcast}

  Causal broadcast is a \textbf{reliable broadcast} that ensures a
  \textbf{specific ordering} among message deliveries.
  
  \vspace{1em}

  \begin{definition}[Uniform reliable broadcast] 
    When a process~A broadcasts a message to all processes of its system $b_A(m)$,
    each correct process~B eventually receives it and delivers it
    $d_B(m)$. URB guarantees validity, uniform agreement, and uniform integrity.
    %% 3 properties:
    % \begin{itemize}
    % \item Validity: If a correct process broadcasts a message, then it
    %   eventually delivers it.
    % \item Uniform Agreement: If a process -- correct or not -- delivers a message,
    %   then all correct processes eventually deliver it.
    % \item Uniform Integrity: A process delivers a message at most once, and only if
    %   it was previously broadcast.
    % \end{itemize}
  \end{definition}

  \begin{minipage}{0.49\textwidth} %% fails to receive a message
    %\begin{center}
      \input{input/figtimelineC.tex}\\
      \small\textit{A process must receive every message}
    %\end{center}
  \end{minipage}~~~
  \begin{minipage}{0.49\textwidth} %% multiple deliveries
    \begin{center}
      \input{input/figtimelineD.tex}
      \small\textit{and deliver each message once}
    \end{center}
  \end{minipage}
  
  
  % \vspace{1em}

  % Reliable broadcast forbids multiple delivery.
  
\end{frame}


\begin{frame}{Possible implementations of URB}
  
  \begin{minipage}{0.49\textwidth}
    % \begin{center}
      \input{input/figtimelineE.tex}
      % \end{center}
  \end{minipage}~
  \begin{minipage}{0.49\textwidth}
    \begin{center}
      \input{input/figtimelineF.tex}
    \end{center}
  \end{minipage}
  \begin{minipage}{0.49\textwidth}
    \vspace{1em}
    Constrain the system topology.\\
    Memory consumption: None \YES{\cmark}\\
    Departures/crashes: \NO{\xmark}
  \end{minipage}~
  \begin{minipage}{0.49\textwidth}
    \vspace{1em}
    Save control information.\\
    Memory consumption: $O(N)$ \NO{\xmark}\\
    Departures/crashes: \YES{\cmark}
  \end{minipage}
  
\end{frame}

\begin{frame}[standout]
  
  How to implement an efficient causal broadcast when they rely on reliable
  broadcast implementations that do not scale in large and dynamic systems?
  
  \vspace{2em}
  \large
  (How to implement an efficient reliable broadcast in large and
  dynamic systems?)

\end{frame}

\begin{frame}[standout] 
  How to forget obsolete control information about broadcast messages?
\end{frame}


\begin{frame}{Proposal: PRC-broadcast}

  PRC-broadcast is an implementation of reliable causal broadcast.

  \vspace{1em}
  
  PRC-broadcast exploits the topology without constraining it. It maintains a
  local data structure the size of which does not monotonically increases.

  \vspace{1em}
  
  This implementation of causal broadcast uses the causal order to improve on
  the underlying reliable broadcast implementation.

\end{frame}


\begin{frame}{Link memory in static systems: \YES{\cmark}}

  \begin{definition}[Link memory]
    A link from Process~A to Process~B remembers among Process~B's delivered
    messages those that will be received from Process~A; and forgets among
    Process~B's delivered messages those that will never be received from
    Process~A.
    $\forall m,\, remember_{BA}(m) \equiv d_B(m) \wedge \neg r_{BA}(m)$
  \end{definition}

  \vspace{1em}

  \only<1>{
    \begin{center}
      \begin{minipage}{0.48\textwidth}
        \begin{center}
          \input{input/figcomplexityA.tex}
        \end{center}
      \end{minipage}
    \end{center}
  }
  \only<2>{
    \begin{center}
      \begin{minipage}{0.48\textwidth}
        \begin{center}
          \input{input/figcomplexityB.tex}
        \end{center}
      \end{minipage}
    \end{center}
  }
  \only<3>{
    \begin{center}
      \begin{minipage}{0.48\textwidth}
        \begin{center}
          \input{input/figcomplexityC.tex}
        \end{center}
      \end{minipage}
    \end{center}
  }
  \only<4>{
    \begin{center}
      \begin{minipage}{0.48\textwidth}
        \begin{center}
          \input{input/figcomplexityD.tex}
        \end{center}
      \end{minipage}
    \end{center}
  }
  \only<5>{
    \begin{center}
      \begin{minipage}{0.48\textwidth}
        \begin{center}
          \input{input/figcomplexityE.tex}
        \end{center}
      \end{minipage}
    \end{center}
  }


  \vspace{1em}

  \begin{theorem}[Link memory forbids multiple delivery]
    Assuming that each link conveys each message at most once, link memory is
    sufficient to forbid multiple delivery.
  \end{theorem}

\end{frame}


% \begin{frame}{Large scale static systems: \YES{\cmark}}
  
%   reliable links + forwarding + link memory = reliable broadcast\\
%   + FIFO links = causal broadcast


%   \begin{minipage}{0.24\textwidth}
%     \centering
%     \input{input/figmemorylinkA.tex}    
%   \end{minipage}
%   \begin{minipage}{0.24\textwidth}
%     \vspace{11pt}
%     \centering
%     \input{input/figmemorylinkB.tex}    
%   \end{minipage}
%   \begin{minipage}{0.24\textwidth}
%     \vspace{11pt}
%     \centering
%     \input{input/figmemorylinkC.tex}    
%   \end{minipage}
%   \begin{minipage}{0.24\textwidth}
%     \vspace{1pt}
%     \centering
%     \input{input/figmemorylinkD.tex}
%   \end{minipage}

%   \vspace{2em}

%   All and only obsolete control information about delivered broadcast messages
%   is removed.

% \end{frame}

\begin{frame}{Link memory in dynamic systems: \NO{\xmark}}
  
  \only<1>{
    \begin{center}
      \begin{minipage}{0.48\textwidth}
        \begin{center}
          
\begin{tikzpicture}

  \small

  \newcommand\X{180/5pt};
  \newcommand\Y{29pt};

  \draw[fill=white] (0*\X, 0*\Y) node{\textbf{A}} +(-5pt, -5pt) rectangle +(5pt, 5pt);

  \draw[->] (-1*\X, -1*\Y) -- node[below, sloped]{$b$} (-5pt, -5pt);
  \draw[->] (-1*\X , 0*\Y) -- (-5pt, 0pt);
  % \draw[->] (-1*\X,  1*\Y) -- (-5pt, 5pt);

  \draw[<-] (1*\X, -1*\Y) -- (5pt, -5pt);
  \draw[<-] (1*\X , 0*\Y) -- (5pt, 0pt);
  \draw[<-] (1*\X,  1*\Y) -- (5pt, 5pt);
  
\end{tikzpicture}
        \end{center}
      \end{minipage}
    \end{center}   
  }
  \only<2>{
    \begin{center}
      \begin{minipage}{0.48\textwidth}
        \begin{center}
          
\begin{tikzpicture}[scale=1.3]

  \small

  \newcommand\X{180/5pt};
  \newcommand\Y{29pt};

  \draw[fill=white] (0*\X, 0*\Y) node{\textbf{A}} +(-5pt, -5pt) rectangle +(5pt, 5pt);

  \draw[->] (-1*\X, -1*\Y) -- (-5pt, -5pt);
  \draw[->] (-1*\X , 0*\Y) -- node[above]{\footnotesize{$[b]$}} (-5pt, 0pt);
  \draw[->, opacity=0] (-1*\X,  1*\Y) -- node[above, sloped]{\footnotesize{$[b]$???}} (-5pt, 5pt);
  
  % \draw[->] (-1*\X,  1*\Y) -- (-5pt, 5pt);

  \draw[<-] (1*\X, -1*\Y) -- node[above, sloped]{$b$} (5pt, -5pt);
  \draw[<-] (1*\X , 0*\Y) -- node[above]{$b$} (5pt, 0pt);
  \draw[<-] (1*\X,  1*\Y) -- node[above, sloped]{$b$} (5pt, 5pt);
  
\end{tikzpicture}
        \end{center}
      \end{minipage}
    \end{center}   
  }
  \only<3->{
    \begin{center}
      \begin{minipage}{0.48\textwidth}
        \begin{center}
          \input{input/figfailsC.tex}\\
          \small\textbf{\textit{should I expect $b$ from the new link?}}
        \end{center}
      \end{minipage}
    \end{center}   
  }
  \only<4->{
    \ \\
    \begin{center}
      \begin{minipage}{0.48\textwidth}
        \begin{center}
          \input{input/figfailsD.tex}\\
          \small\textit{infinitely increasing memory consumption}
        \end{center}
      \end{minipage}
      \begin{minipage}{0.48\textwidth}
        \begin{center}
          
\begin{tikzpicture}[scale=1.3]

  \small

  \newcommand\X{180/5pt};
  \newcommand\Y{29pt};

  \draw[fill=white] (0*\X, 0*\Y) node{\textbf{A}} +(-5pt, -5pt) rectangle +(5pt, 5pt);

  \draw[->] (-1*\X, -1*\Y) -- (-5pt, -5pt);
  \draw[->] (-1*\X , 0*\Y) -- node[above]{b \sout{\footnotesize{$[b]$}}} (-5pt, 0pt);
  \draw[->, color=blue] (-1*\X,  1*\Y) -- node[above, sloped]{\NO{$b$}} (-5pt, 5pt);

  \draw[<-] (1*\X, -1*\Y) -- (5pt, -5pt);
  \draw[<-] (1*\X , 0*\Y) -- (5pt, 0pt);
  \draw[<-] (1*\X,  1*\Y) -- (5pt, 5pt);
  
\end{tikzpicture}\\
          \small\textit{Multiple delivery and cascading effect over the whole system}
        \end{center}
      \end{minipage}
    \end{center}
  }
  



  % \begin{minipage}{0.24\textwidth}
  %   \centering
  %   \input{input/figmemorylinkfailsA.tex}    
  % \end{minipage}
  % \begin{minipage}{0.24\textwidth}
  %   \centering
  %   \input{input/figmemorylinkfailsB.tex}    
  % \end{minipage}
  % \begin{minipage}{0.24\textwidth}
  %   \centering
  %   \input{input/figmemorylinkfailsC.tex}    
  % \end{minipage}
  % \begin{minipage}{0.24\textwidth}
  %   \centering
  %   \input{input/figmemorylinkfailsD.tex}
  % \end{minipage}
  
  % \vspace{1.5em}

  % Adding a link adds uncertainty. Process~C does not know messages it should
  % expect from Process~B. It mistakes the message $a$ for a new one.

  % Not only it delivers $a$ multiple times, but this has cascading effects over
  % the whole system.

\end{frame}


\begin{frame}[standout]
  How can processes initialize link memory?
\end{frame}

\begin{frame}{Order to the help of link memory initialization}

  \begin{center}
    \input{input/figtimelineproof.tex}
  \end{center}

  \vspace{-1em}

  \begin{minipage}{0.325\textwidth}
    \setbeamercolor{block title}{use=structure,fg=black,bg=green!20}
    \setbeamercolor{block body}{use=structure,fg=black,bg=green!10}
    \begin{block}{Lemma (Buffer $B_\alpha$)}
      $B_\alpha$ contains $\mathcal{A}_2$ and $\mathcal{B}_2$.
    \end{block}
  \end{minipage}
  \begin{minipage}{0.325\textwidth}
    \setbeamercolor{block title}{use=structure,fg=black,bg=blue!20}
    \setbeamercolor{block body}{use=structure,fg=black,bg=blue!10}
    \begin{block}{Lemma (Buffer $B_\beta$)}
      $B_\beta$ contains $\mathcal{B}_2$ and $\mathcal{A}_3$.
    \end{block}
  \end{minipage}
  \begin{minipage}{0.325\textwidth}
    \setbeamercolor{block title}{use=structure,fg=black,bg=purple!20}
    \setbeamercolor{block body}{use=structure,fg=black,bg=purple!10}
    \begin{block}{Lemma (Buffer $B_\pi$)}
      $B_\pi$ contains $\mathcal{A'}_3$ and $\mathcal{B}_3$ with 
      $\mathcal{A}'_3 \subseteq \mathcal{A}_3$.
    \end{block}
  \end{minipage}
  
  \setbeamercolor{block title}{use=structure,fg=black,bg=yellow!20}
  \setbeamercolor{block body}{use=structure,fg=black,bg=yellow!10}
  \begin{block}{Lemma $B_\alpha$, $B_\beta$, $B_\pi$ initialize link memory}
    The memory of a new link becomes correct at receipt of $B_\beta$. \\
    To deliver:
    $B_\beta \setminus B_\alpha \setminus B_\pi = (\mathcal{B}_2 \cup
    \mathcal{A}_3) \setminus (\mathcal{A}_2 \cup \mathcal{B}_2) \setminus
    (\mathcal{A}_3' \cup \mathcal{B}_3) = \mathcal{A}_3 \setminus
    \mathcal{A}_3'$\\
    To expect:
    $B_\pi \setminus B_\beta = (\mathcal{A}_3' \cup \mathcal{B}_3) \setminus
    (\mathcal{B}_2 \cup \mathcal{A}_3) = \mathcal{B}_3$

  \end{block}   
    
\end{frame}


\begin{frame}{Implementation}

  \only<1>{
    \begin{center}
      
\begin{tikzpicture}[scale=1]

  \small
  
  \newcommand\X{1.9*\columnwidth/16pt};
  \newcommand\YA{0pt};
  \newcommand\YB{-60pt};
  
  \newcommand\YSPACE{13pt};

  \draw[->, thick](0*\X, \YA) -- (8*\X, \YA);
  \draw[->, thick](0*\X, \YB) -- (8*\X, \YB);
  
  \draw[fill=white] (0*\X, \YA) node{\textbf{\textup{B}}} +(-5pt, -5pt) rectangle +(5pt, 5pt);
  \draw[fill=white] (0*\X, \YB) node{\textbf{\textup{C}}} +(-5pt, -5pt) rectangle +(5pt, 5pt);
  
  \draw[->] (0*\X, \YA + 1* \YSPACE)% node[above right]{$\mathcal{A}$}
  -- node[below]{$\varnothing$} (2*\X, \YA + \YSPACE);
  \draw[color=fg!30, densely dashed,->] ( 2*\X, 1*\YA )
  -- (3*\X, 1*\YB);
  \draw[densely dashed,->] ( 2*\X, 1*\YA )
  -- node[sloped, above]{$\alpha$} (2.3*\X, 0.3*\YB);

  \draw[color=fg!30,->] (0*\X, \YB - 1* \YSPACE)
  -- (3*\X, \YB - \YSPACE);  
  \draw[->] (0*\X, \YB - 1* \YSPACE)
  -- node[above]{$\varnothing$}
  (2.3*\X, \YB - \YSPACE);
  

  \draw[color=fg!30, densely dashed,->] ( 3*\X, \YB ) 
  -- node[sloped, above]{$\beta$} (4*\X, \YA);


  \draw[color=fg!30, <->] ( 2*\X, \YA + 1* \YSPACE )
  -- node[below]{$\mathcal{B}_1 \cup \mathcal{A}_2$}
  (4*\X, \YA + 1* \YSPACE);
  
  \draw[color=fg!30, densely dashed,->] ( 4*\X, \YA )
  -- node[sloped, above]{$\pi$} (5*\X, \YB);

  \draw[<->,very thick, color=green!30] (3*\X, \YB - 1*\YSPACE) --
  node[below]{$\bm{B_\alpha}$}
  node[above]{$\mathcal{A}_2 \cup \mathcal{B}_2$}
  (5*\X, \YB - 1*\YSPACE);

  \draw[color=fg!30, densely dashed,->] ( 5*\X, \YB ) 
  -- node[sloped, above]{$\rho$} (6*\X, \YA);

  \draw[<->,very thick, color=blue!20] (4*\X, \YA +1*\YSPACE)
  -- node[above]{$\bm{B_\beta}$}
  node[below]{$\mathcal{B}_2 \cup \mathcal{A}_3$} (6*\X, \YA + 1*\YSPACE);


  \draw[->, very thick, color=yellow!30] ( 6*\X, \YA )
  -- node[sloped, above]{$\bm{B_\beta}$} (7*\X, \YB);

  
  \draw[<->,very thick, color=purple!20] (5*\X, \YB - 1*\YSPACE )
  -- node[below]{$\bm{B_\pi}$}
  node[above]{$\mathcal{A}_3' \cup \mathcal{B}_3$ with
    $\mathcal{A}_3' \subseteq \mathcal{A}_3$} 
  (7*\X, \YB - 1*\YSPACE);

\end{tikzpicture}


    \end{center}
  }
  \only<2>{
    \begin{center}
      \input{input/timelinesproof/figB.tex}
    \end{center}
  }
  \only<3>{
    \begin{center}
      \input{input/timelinesproof/figC.tex}
    \end{center}
  }
  \only<4>{
    \begin{center}
      \input{input/timelinesproof/figD.tex}
    \end{center}
  }
  \only<5>{
    \begin{center}
      \input{input/timelinesproof/figE.tex}
    \end{center}
  }
  \only<6>{
    \begin{center}
      \input{input/timelinesproof/figF.tex}
    \end{center}
  }
  \only<7>{
    \begin{center}
      
\begin{tikzpicture}[scale=1]

  \small
  
  \newcommand\X{1.9*\columnwidth/16pt};
  \newcommand\YA{0pt};
  \newcommand\YB{-60pt};
  
  \newcommand\YSPACE{13pt};

  \draw[->, thick](0*\X, \YA) -- (8*\X, \YA);
  \draw[->, thick](0*\X, \YB) -- (8*\X, \YB);
  
  \draw[fill=white] (0*\X, \YA) node{\textbf{\textup{B}}} +(-5pt, -5pt) rectangle +(5pt, 5pt);
  \draw[fill=white] (0*\X, \YB) node{\textbf{\textup{C}}} +(-5pt, -5pt) rectangle +(5pt, 5pt);
  
  \draw[->] (0*\X, \YA + 1* \YSPACE)
  -- node[below]{$\varnothing$} (2*\X, \YA + \YSPACE);
  \draw[densely dashed,->] ( 2*\X, 1*\YA )
  -- node[sloped, above]{$\alpha$} (3*\X, 1*\YB);

  \draw[->] (0*\X, \YB - 1* \YSPACE)
  -- node[above]{$\varnothing$}
  (3*\X, \YB - \YSPACE);  

  \draw[densely dashed,->] ( 3*\X, \YB ) 
  -- node[sloped, above]{$\beta$} (4*\X, 0*\YB);

  \draw[<->] ( 2*\X, \YA + 1* \YSPACE )
  -- node[below]{$\varnothing$}
  (4*\X, \YA + 1* \YSPACE);
  
  \draw[densely dashed,->] ( 4*\X, \YA )
  -- node[sloped, above]{$\pi$} (5*\X, 1*\YB);

  \draw[<->,very thick, color=green!80!black!80] (3*\X, \YB - 1*\YSPACE) --
  node[above]{$c_1, c_2$}
  node[below]{$\bm{B_\alpha}$}
  (5*\X, \YB - 1*\YSPACE);

  \draw[color=fg!30, densely dashed,->] ( 5*\X, \YB ) 
  -- (6*\X, \YA);  
  \draw[densely dashed,->] ( 5*\X, \YB ) 
  -- node[sloped, above]{$\rho$} (5.4*\X, 0.6*\YB);

  \draw[<->,very thick, color=blue!20] (4*\X, \YA +1*\YSPACE)
  -- node[above]{$\bm{B_\beta}$}
  (6*\X, \YA + 1*\YSPACE);
  \draw[<->,very thick, color=blue!80!black!80] (4*\X, \YA +1*\YSPACE)
  -- node[below]{$c_1, b_1, b_2$}
  (5.4*\X, \YA + 1*\YSPACE);


  \draw[->, very thick, color=yellow!30] ( 6*\X, \YA )
  -- node[sloped, above]{$\bm{B_\beta}$}
  (7*\X, \YB);

  \draw[<->,very thick, color=purple!20] (5*\X, \YB - 1*\YSPACE )
  -- node[below]{$\bm{B_\pi}$}
  (7*\X, \YB - 1*\YSPACE);
  \draw[<->,very thick, color=purple!80!black!80] (5*\X, \YB - 1*\YSPACE )
  -- node[above]{$b_1, c_3$}
  (5.4*\X, \YB - 1*\YSPACE);  


\end{tikzpicture}

    \end{center}
  }
  \only<8>{
    \begin{center}
      \input{input/timelinesproof/figH.tex}
    \end{center}
  }
  \only<9>{
    \begin{center}
      \input{input/timelinesproof/figI.tex}
    \end{center}
  }
  
  \vspace{-1em}
  
  \only<1>{
    \begin{center}
      \input{input/figsolveA.tex}
    \end{center}
  }
  \only<2>{
    \begin{center}
      \input{input/figsolveB.tex}
    \end{center}
  }
  \only<3>{
    \begin{center}
      \input{input/figsolveC.tex}
    \end{center}
  }
  \only<4>{
    \begin{center}
      \input{input/figsolveD.tex}
    \end{center}
  }
  \only<5>{
    \begin{center}
      \input{input/figsolveE.tex}
    \end{center}
  }
  \only<6>{
    \begin{center}
      \input{input/figsolveF.tex}
    \end{center}
  }
  \only<7>{
    \begin{center}
      \input{input/figsolveG.tex}
    \end{center}
  }
  \only<8>{
    \begin{center}
      \input{input/figsolveH.tex}
    \end{center}
  }
  \only<9>{
    \begin{center}
      
\begin{tikzpicture}[scale=1.3]
  

  \small
  
  \newcommand\X{190/5pt};
  \newcommand\Y{29pt};

  \draw[fill=white, color=white, opacity=0] (-2*\X, -2*\Y) rectangle (4*\X, 1*\Y);
  
  
  \draw[fill=white] (0*\X, 0*\Y) node{\textbf{A}} +(-5pt, -5pt) rectangle +(5pt, 5pt);
  % \draw (-5+0*\X, 0*\Y) node[left]{$E: \{a:2\}$};
  \draw[fill=white] (1*\X, -1*\Y) node{\textbf{B}} +(-5pt, -5pt) rectangle +(5pt, 5pt);
  % \draw (1*\X, 5-1*\Y) node[above]{$\rho$};
  \draw (5+ 1*\X, -5-1*\Y) node[below right]{\phantom{$B_\beta: [c_1,\,b_1,\,b_2,\,c_2]$}};
  % \draw (1*\X, -5-1*\Y) node[below]{$E: \varnothing$\vphantom{$\{$}};
  \draw[fill=white] (2*\X,  0*\Y) node{\textbf{C}} +(-5pt, -5pt) rectangle +(5pt, 5pt);
  % \draw(2*\X, 5+0*\Y) node[above]{$\pi$};
  \draw (5+2*\X, 0*\Y) node[right, align=left]{\BA{$B_\alpha: \{c_1,\, c_2\}$}\\\BP{$B_\pi: \{ b_1,\, c_3 \}$}\\\BB{$B_\beta: [c_1,\,b_1,\,b_2,\,c_2]$}};
  
  \draw[->](5+0*\X, 0*\Y) -- node[sloped, above] {$\bm{c_3}$ $b_2$} (-5+1*\X, -1*\Y); %% A->B
  \draw[<-](5+0*\X, -5+0*\Y) -- node[sloped, below]{~~~~~ $c_2$} (-5+1*\X, -5-1*\Y); %% A<-B
  
  \draw[->](5+0*\X, 5+0*\Y) -- node[above]{$c_3$ $\bm{b_2}$ ~~~~~~} (-5+2*\X, 5+0*\Y); % A->C
  \draw[<-](5+0*\X,  1.25+ 0*\Y) -- (-5+2*\X,  1.25+ 0*\Y); % A<-C
  
  \draw[->, dashed](5+1*\X, -1*\Y) -- node[below right, sloped, color=red]{$[???]$} (-5+2*\X, 0*\Y); %% B<-C
  % \draw[->, dashed](5+1*\X, -5-1*\Y) -- (-5+2*\X, -5+0*\Y); %% B->C
\end{tikzpicture}

    \end{center}
  }

\end{frame}


% \begin{frame}{Control messages with causal order implement link memory}
  
%   \begin{adjustwidth}{-1.5em}{-1.5em}

%   \begin{minipage}{0.35\textwidth}
%     \raggedright
%     \vspace{-7pt}
%     \input{input/figsolveA.tex}
%   \end{minipage}
%   \begin{minipage}{0.36\textwidth}
%     \raggedright
%     \vspace{-14pt}
%     \input{input/figsolveB.tex}
%   \end{minipage}
%   \begin{minipage}{0.28\textwidth}
%     \raggedright
%     \input{input/figsolveC.tex}
%   \end{minipage}

%   \begin{minipage}{0.34\textwidth}
%     \raggedright
%     \input{input/figsolveD.tex}
%   \end{minipage}
%   \begin{minipage}{0.34\textwidth}
%     \raggedright
%     \vspace{9pt}
%     \input{input/figsolveE.tex}
%   \end{minipage}
%   \hspace{5pt}
%   \begin{minipage}{0.31\textwidth}
%     \raggedright
%     \input{input/figsolveF.tex}
%   \end{minipage}

%   \begin{minipage}{0.53\textwidth}
%     \raggedleft
%     \input{input/figsolveG.tex}
%   \end{minipage}
%   \begin{minipage}{0.45\textwidth}
%     \centering
%     \input{input/figsolveH.tex}        
%   \end{minipage}
    

%   \end{adjustwidth}

% \end{frame}

\begin{frame}{Disambiguation: to deliver, to expect, to ignore}
  
  \begin{adjustwidth}{-1.5em}{-1.5em}
    \begin{center}
    
\begin{tikzpicture}[scale=1.3]
  

  \small
  
  \newcommand\X{190/5pt};
  \newcommand\Y{29pt};

  \draw[fill=white, color=white, opacity=0] (-2*\X, -2*\Y) rectangle (4*\X, 1*\Y);
  
  
  \draw[fill=white] (0*\X, 0*\Y) node{\textbf{A}} +(-5pt, -5pt) rectangle +(5pt, 5pt);
  % \draw (-5+0*\X, 0*\Y) node[left]{$E: \{a:2\}$};
  \draw[fill=white] (1*\X, -1*\Y) node{\textbf{B}} +(-5pt, -5pt) rectangle +(5pt, 5pt);
  % \draw (1*\X, 5-1*\Y) node[above]{$\rho$};
  \draw (5+ 1*\X, -5-1*\Y) node[below right]{\phantom{$B_\beta: [c_1,\,b_1,\,b_2,\,c_2]$}};
  % \draw (1*\X, -5-1*\Y) node[below]{$E: \varnothing$\vphantom{$\{$}};
  \draw[fill=white] (2*\X,  0*\Y) node{\textbf{C}} +(-5pt, -5pt) rectangle +(5pt, 5pt);
  % \draw(2*\X, 5+0*\Y) node[above]{$\pi$};
  \draw (5+2*\X, 0*\Y) node[right, align=left]{\BA{$B_\alpha: \{c_1,\, c_2\}$}\\\BP{$B_\pi: \{ b_1,\, c_3 \}$}\\\BB{$B_\beta: [c_1,\,b_1,\,b_2,\,c_2]$}};
  
  \draw[->](5+0*\X, 0*\Y) -- node[sloped, above] {$\bm{c_3}$ $b_2$} (-5+1*\X, -1*\Y); %% A->B
  \draw[<-](5+0*\X, -5+0*\Y) -- node[sloped, below]{~~~~~ $c_2$} (-5+1*\X, -5-1*\Y); %% A<-B
  
  \draw[->](5+0*\X, 5+0*\Y) -- node[above]{$c_3$ $\bm{b_2}$ ~~~~~~} (-5+2*\X, 5+0*\Y); % A->C
  \draw[<-](5+0*\X,  1.25+ 0*\Y) -- (-5+2*\X,  1.25+ 0*\Y); % A<-C
  
  \draw[->, dashed](5+1*\X, -1*\Y) -- node[below right, sloped, color=red]{$[???]$} (-5+2*\X, 0*\Y); %% B<-C
  % \draw[->, dashed](5+1*\X, -5-1*\Y) -- (-5+2*\X, -5+0*\Y); %% B->C
\end{tikzpicture}

    \end{center}

    % \vspace{em}

    \input{input/figsolveJ.tex}
  \end{adjustwidth}

\end{frame}


\begin{frame}{Complexity: a new trade-off}

  \begin{table}
    \begin{center}
      
% \setlength{\tabcolsep}{4pt} % General space between cols (6pt standard)

%\scriptsize

\begin{tabularx}{0.90\columnwidth}{@{}cccc@{}}
  \makecell{message\\overhead} &  \makecell{delivery\\execution time} & \makecell{local space\\consumption} & \makecell{\color{fg!30}{\# control messages}\\\color{fg!30}{per added link}} \\ \hline
  $O(1)$ & $O(|Q_i|)$ & $\mathbf{O(|Q_i|\cdot M)}$ & \color{fg!30}{$\mathbf{6}$ to $\mathbf{4\cdot |P|^2}$} \\
%  \bottomrule
\end{tabularx}

%%% Local Variables:
%%% mode: latex
%%% TeX-master: "../paper"
%%% End:

    \end{center}
  \end{table}

  \vspace{1em}

  \begin{minipage}{0.48\textwidth}
    \begin{center}
      \input{input/figcomplexityA.tex}
    \end{center}
  \end{minipage}\hfill
  \begin{minipage}{0.48\textwidth}
    \begin{center}
      \input{input/figcomplexityB.tex}
    \end{center}
  \end{minipage}
  
  \vspace{1em}
  
  We can achieve ${(|Q_i| - 1) \cdot 2\cdot N}\over{2}$ in space consumed. Where
  ${1}\over{2}$ is a switch between ``already received'' and ``expected''
  tags. And $2\cdot N$ is a vector of intervals instead of all messages.
  
  \vspace{1em}

  Most importantly, this structure is \textbf{not grow-only}.

\end{frame}

\begin{frame}{Overhead: depends on the overlay network}

  \begin{table}
    \begin{center}
      
% \setlength{\tabcolsep}{4pt} % General space between cols (6pt standard)

%\scriptsize

\begin{tabularx}{0.90\columnwidth}{@{}cccc@{}}
  \makecell{\color{fg!30}{message}\\\color{fg!30}{overhead}} &  \makecell{\color{fg!30}{delivery}\\\color{fg!30}{execution time}} & \makecell{\color{fg!30}{local space}\\\color{fg!30}{consumption}} & \makecell{\# control messages\\per added link} \\ \hline
  \color{fg!30}{$O(1)$} & \color{fg!30}{$O(|Q_i|)$} & \color{fg!30}{$\mathbf{O(|Q_i|\cdot M)}$} & $\mathbf{6}$ to $\NO{\mathbf{4\cdot |P|^2}}$ \\
%  \bottomrule
\end{tabularx}

%%% Local Variables:
%%% mode: latex
%%% TeX-master: "../paper"
%%% End:

    \end{center}
  \end{table}
  
  \vspace{1em}
  
  \begin{center}
    \input{input/figneighbor2neighbor.tex}
  \end{center}

  %\vspace{0.3em}

  \begin{center}
    \begin{table}
      \begin{center}
        %\setlength{\tabcolsep}{3pt} % General space between cols (6pt standard)
\footnotesize

\begin{tabularx}{0.95\columnwidth}{@{}Xc@{}}
%  \toprule
  & \makecell{number of control messages\\per new link} \\
  \hline\hline
%  objective & \cmark & $?<O(N)$ & $O(N)\leq \, ?$ & $? \leq O(W.N)$ \\ \hline\hline

  \textbf{best-case} & $6$ \\ 
  \textbf{neighbor-to-neighbor connection (gossip)} & $8$ \\ 
  \textbf{routing (DHT)} & $O(4 \cdot \log P)$  \\
  \textbf{worst-case (broadcast + complete graph)} & $\NO{4 \cdot P^2}$ 
%  \bottomrule
\end{tabularx}

%%% Local Variables:
%%% mode: latex
%%% TeX-master: "../paper"
%%% End:

      \end{center}
    \end{table}
  \end{center}
  
  \vspace{1em}

  Most importantly, the number of messages can be \textbf{small} and
  \textbf{constant}.


\end{frame}


\begin{frame}{Experiments: setup}
  
  \textbf{Goal:} Show the trade-off proposed by PRC-broadcast experimentally.

  \vspace{1em}

  A random peer-sampling protocol builds the overlay network. The topology has
  properties close to those of random graphs.

  \vspace{1em}
  
  The system is very dynamic. Processes periodically exchange their outgoing
  links.
  
  \vspace{1em}
  
  Links are bidirectional. Each new bidirectional link still costs 8 control
  messages.
  
  
  \input{input/figtimelinebirpcbroadcast.tex}
  
\end{frame}


\begin{frame}{Experiments: new complexity in local space consumed}

  \hspace{-1em}\includegraphics[width=1.0\textwidth]{img/overhead.eps}

\end{frame}


\begin{frame}{Experiments: small traffic overhead}

  \hspace{-1em}\includegraphics[width=1.0\textwidth]{img/controlmessages.eps}
    
\end{frame}


\begin{frame}{Conclusion}
  
  PRC-broadcast constitutes a new trade-off for causal broadcast.

  \vspace{2em}

  \begin{table}
    \begin{center}
      
\setlength{\tabcolsep}{4pt} % General space between cols (6pt standard)

\small

\begin{tabularx}{0.98\columnwidth}{@{}Xcccc@{}}
  & \makecell{message\\overhead} &  \makecell{delivery\\execution time} & \makecell{local space\\consumption} & \makecell{\# control messages\\per added link} \\ \cmidrule{2-5}
  vector-based R-broadcast & $O(1)$ & $O(1)$ & $O(\NO{N})$ & $0$ \\
  vector-based C-broadcast & $O(\NO{N})$ & $O(W.\NO{N})$ & $O(\NO{N}+W.\NO{N})$ & $0$ \\ 
  \footnotesize{PC-broadcast} & $O(1)$ & $O(1)$ & $O(\NO{N})$ & $3$ to \NO{$2P^2$} \\ \hline\hline
  \scriptsize{PRC-broadcast} & $O(1)$ & $O(Q_i)$ & $\mathbf{O(Q_i.M)}$ & $\mathbf{6}$ to $\NO{\mathbf{4P^2}}$ \\
%  \bottomrule
\end{tabularx}

%%% Local Variables:
%%% mode: latex
%%% TeX-master: "../paper"
%%% End:

    \end{center}
  \end{table}

  \vspace{2em}

  \faLightbulbO~Fun fact! This implementation of causal broadcast even makes an
  efficient implementation for reliable broadcast!

\end{frame}


\begin{frame}[standout]

  To be continued\ldots Retrieving partial order out of flattened orders.
  
  \vspace{2em}
  
  \begin{center}
    \input{input/figflattened.tex}
  \end{center}
  
\end{frame}


\begin{frame}[standout]

  \vspace{6em}
  
  Thanks!

  \vspace{5em}

  \small
  \faEnvelope ~ brice.nedelec@ls2n.fr

\end{frame}

\end{document}
